%!TEX program = xelatex
\documentclass[12pt]{article}
\usepackage{amsthm}
\usepackage{amsmath}
\usepackage{amsfonts}
\usepackage{mathtools}
\usepackage{array}

\newcommand{\N}{\ensuremath{\mathbb{N}}}
\newcommand{\R}{\ensuremath{\mathbb{R}}}

\begin{document}

\section*{Problem 10}

\begin{proof}
  \begin{align*}
    \lim_{x \to 1^-}\frac{g(x)-g(1)}{x-1} = \lim_{x \to 1^-} 3(x+1) = 6.
  \end{align*}
  If we want $g(x)$ to be differentiable at $x=1$, we need $\lim_{x \to 1^+}\frac{g(x)-g(1)}{x-1}=6$ as well. For any $x>1$,
  \begin{align*}
    \frac{g(x)-g(1)}{x-1} &= \frac{a+bx - 3}{x-1} = \frac{bx +(a-3)}{x-1}.
  \end{align*}
  If $b\not=0$, then we can factor out $b$ and obtain
  \begin{align*}
    \frac{g(x)-g(1)}{x-1} &= \frac{b(x +\frac{a-3}{b})}{x-1}.
  \end{align*}
  If either $b=0$, or $b\not=0$ but $\frac{a-3}{b} \not=-1$, then the difference quotient does not converge as $x \rightarrow 1$. Therefore, $\frac{a-3}{b}=-1 \implies   a+b=3$. In this case,
  \begin{align*}
    \lim_{x \to 1^+}\frac{g(x)-g(1)}{x-1} = \lim_{x \to 1^+} \frac{b(x-1)}{x-1} = b.
  \end{align*}
  Therefore, we need to let $b=6$ and $a=3-b=-3$ for this function to be differentiable at $x=1$.
\end{proof}


\section*{Problem 17}

\begin{proof}
  For any $x\not=0$, we have
  \begin{align*}
    \frac{f(ax)-f(bx)}{cx} &= \frac{1}{c} \cdot \frac{f(ax)-f(bx)}{x}\\
                           &= \frac{1}{c} \cdot \frac{f(ax)-f(0)+f(0)-f(bx)}{x}\\
                           &= \frac{1}{c} \left[ \frac{f(ax)-f(0)}{x} - \frac{f(bx)- f(0)}{x} \right]
  \end{align*}
  Therefore,
  \begin{align*}
   \lim_{x \to 0} \frac{f(ax)-f(bx)}{cx}
   &= \lim_{x \to 0} \frac{1}{c} \left[ \frac{f(ax)-f(0)}{x} - \frac{f(bx)- f(0)}{x} \right] \\
   &= \lim_{x \to 0} \frac{1}{c} \left[ \lim_{x \to 0}\frac{f(ax)-f(0)}{x} - \lim_{x \to 0}\frac{f(bx)- f(0)}{x} \right] \\
   &= \frac{1}{c} \left[ \lim_{x \to 0}\frac{f(ax)-f(0)}{x} - \lim_{x \to 0}\frac{f(bx)- f(0)}{x} \right].
  \end{align*}
  We will first prove that $\lim_{x \to 0}\frac{f(ax)-f(0)}{x} =af^\prime(0)$. There are two cases:
  \begin{enumerate}
    \item If $a=0$, then for any $x \not=0$,
      \begin{align*}
        \frac{f(ax)-f(0)}{x} = \frac{f(0)-f(0)}{x} = 0.
      \end{align*}
      So $\lim_{x \to 0}\frac{f(ax)-f(0)}{x} = \lim_{x \to 0} 0 =af^\prime(0) $.
    \item If $a\not=0$, then we can define $h(x)=ax$, which is continuous at $x=0$. Since $a\not=0$, if $x \not= 0$, $h(x)=ax \not = h(0)$. Then, by Exercise 6, we obtain
      \begin{align*}
        \lim_{x \to 0}\frac{f(h(x))-f(h(0))}{h(x)- h(0)} = f^\prime(h(0)),
      \end{align*}
      which is
      \begin{align*}
        \lim_{x \to 0}\frac{f(ax)-f(0)}{ax} = f^\prime(0).
      \end{align*}
      Therefore,
      \begin{align*}
        \lim_{x \to 0}\frac{f(ax)-f(0)}{x} &= \lim_{x \to 0}a \cdot \frac{f(ax)-f(0)}{ax}\\
                                           &= \lim_{x \to 0}a \lim_{x \to 0}\cdot \frac{f(ax)-f(0)}{ax}\\
                                           &= af^\prime(0).
      \end{align*}
  \end{enumerate}
  Similarly, we can obtain $\lim_{x \to 0}\frac{f(bx)-f(0)}{x} =bf^\prime(0)$. Hence,
  \begin{align*}
   \lim_{x \to 0} \frac{f(ax)-f(bx)}{cx}
   &= \frac{1}{c} \left[ \lim_{x \to 0}\frac{f(ax)-f(0)}{x} - \lim_{x \to 0}\frac{f(bx)- f(0)}{x} \right] \\
   &= \frac{1}{c} \left[ a f^\prime(0)  - b f^\prime(0) \right] \\
   &= \left[ \frac{a-b}{c} \right]  f^\prime(0)
  \end{align*}
\end{proof}


\section*{Problem 5}

\begin{proof}
  By contradiction; suppose $f^{-1}$ is differentiable at $f(x_0)$, then because $f$ is differentiable at $x_0$, by the Chain Rule,
  \begin{align*}
    \left[ f^{-1}(f(x_0)) \right]^\prime &= \left( f^{-1} \right)^\prime \left(f(x_0)\right) \cdot f^\prime(x_0)\\
                                         &= \left( f^{-1} \right)^\prime \left(f(x_0)\right) \cdot 0\\ &= 0.
  \end{align*}
  However, since $f^{-1} \left( f(x) \right) =x$ for any $x \in I$ and $(x)^\prime = 1 \not=0$. We have reached a contradiction. Therefore, $f^{-1}$ is not differentiable at $x_0$.
\end{proof}

\section*{Problem 8}

\begin{proof}
  Since the sequence $\{x_n\}$ is strictly increasing and bounded, by the Monotone Convergence Theorem, $\{x_n\}$ converges to some number $x_0 = \sup \{x_n \,|\, n \in \N\}$. Because $x_n$ is strictly increasing, and $x_0 \geq x_n$ for any $n \in \N$, we can prove that $x_n < x_0$ for any $n \in \N$.

  By contradiction; suppose there is some $k \in \N$ such that $x_k = x_0$, then $x_{k+1} > x_k = x_0$, which contradicts the fact that $x_0$ is the supremum of $\{x_n\}$. Therefore, for any $n \in \N$, $x_n < x_0$.

  Since $f:\R \rightarrow \R$ is differentiable, $f$ is also continuous, so the sequence $f(x_n)$ converges to $f(x_0)$ as well. Because $\{f(x_n)\}$ is monotone increasing and convergent, $f(x_n)$ is bounded above by the Monotone Convergence Theorem. We can prove that the upper bound is $f(x_0)$, so that $f(x_n) \leq f(x_0)$ for any $n \in \N$.

  By contradiction; suppose there is some $k \in \N$ such that $f(x_k) > f(x_0)$. Then, by assumption, for any natural number $m \geq k$, $f(x_m) \geq f(x_k)$. However, since $f(x_n)$ is convergent, for any $\varepsilon > 0$, there exists some $N\in \N$ such that if $n \geq N$
  \begin{align*}
    |f(x_n) - f(x_0)| < \varepsilon \implies f(x_n)< f(x_0) + \varepsilon.
  \end{align*}
  Set $\varepsilon = f(x_k) - f(x_0) > 0$, then for any natural number $m \geq k$,
  \begin{align*}
    f(x_m) \geq f(x_k) = f(x_0) + \varepsilon
  \end{align*}
  We have reached a contradiction because for this particular $\varepsilon$ there could not exist any $N \in \N$ such that $n \geq N \implies f(x_n)< f(x_0)+\varepsilon$.
  Therefore, $f(x_0) = \sup \{f(x_n) \,|\, x \in \N\}$.

  Hence, for any $x \not=x_0$, because $x_n < x_0$, the difference quotient
  \begin{align*}
    \frac{f(x_n)-f(x_0)}{x_n-x_0}
  \end{align*}
  is always defined. Also, since $f(x_n) \leq f(x_0)$,
  \begin{align*}
    \frac{f(x_n)-f(x_0)}{x_n-x_0} \geq 0
  \end{align*}
  for any $n \in \N$.

  Because $f$ is differentiable, the difference quotient converges to some nonnegative number, so
  \begin{align*}
    f^\prime(x_0) = \lim_{n \to \infty} \frac{f(x_n)-f(x_0)}{x_n-x_0} \geq 0.
  \end{align*}
\end{proof}


\end{document}
